\documentclass[a4paper]{article}

\usepackage{url}
\usepackage{graphicx}
\usepackage{enumerate}
\usepackage{amsmath}
\usepackage{float}
\usepackage{longtable}
\usepackage{fullpage}
\usepackage{pstricks}
\usepackage{tikz}
\usepackage[absolute]{textpos}
\usepackage{import}
\usepackage{subfigure}

\title{G53NMD Individual Assignment}
\author{Robert J. Golding (rjg08u)} \date{\today}

% Dutch style paragraph formatting
\setlength{\parskip}{1.3ex plus 0.2ex minus 0.2ex}
\setlength{\parindent}{0pt}

\begin{document}
    \maketitle

    \begin{quote}
        Social media on a mobile platform is the pinnacle of excellence of new
        media. Discuss with examples the validity of this statement and how these
        technologies have changed the nature of design.
    \end{quote}

    To understand what Social Media is (and also what it is \emph{not}), one
    must first get a feel for the related topics: Web 2.0 and User Generated
    Content. \cite{kaplan2010}

    Web 2.0 is a term that was first used in 2004 to describe a new way in
    which developers and end-users were starting to use the internet. It refers
    to the move away from the traditional model whereby content was created and
    published by individuals, remaining relatively static from that point on.
    Instead, the web became a collaborative platform upon which content could
    be continuously modified by users themselves.

    Though Web 2.0 does not refer to any particular technical ``update'' of the
    web itself, there are certain underlying technologies that are required for
    it to function. Among these are Adobe Flash (allowing videos, audio, and
    other forms of multimedia to web pages), and RSS (Really Simple
    Syndication, a method for retrieving content that is updated frequently,
    such as blog entries or news headlines).

    \bibliographystyle{alpha}
    \bibliography{references}

\end{document}
