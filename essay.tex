\documentclass[a4paper,11pt]{article}

\usepackage{url}
\usepackage{graphicx}
\usepackage{enumerate}
\usepackage{amsmath}
\usepackage{float}
\usepackage{longtable}
%\usepackage{fullpage}
\usepackage{pstricks}
\usepackage{tikz}
\usepackage[absolute]{textpos}
\usepackage{import}
\usepackage{subfigure}
\usepackage{setspace}

\title{G53NMD Individual Assignment}
\author{Robert J. Golding (rjg08u)} \date{\today}

% Dutch style paragraph formatting
\setlength{\parskip}{1.3ex plus 0.2ex minus 0.2ex}
\setlength{\parindent}{0pt}

%\doublespacing
\onehalfspacing

\begin{document}
    \maketitle

    \begin{quote}
        \emph{Social media on a mobile platform is the pinnacle of excellence
        of new media. Discuss with examples the validity of this statement and
        how these technologies have changed the nature of design.}
    \end{quote}

    \section{Introduction}

    To understand what Social Media is (and also what it is \emph{not}), one
    must first get a feel for the related topics: Web 2.0 and User Generated
    Content. \cite{kaplan2010}

    Web 2.0 is a term that was first used in 2004 to describe a new way in
    which developers and end-users were starting to use the internet. It refers
    to the move away from the traditional model whereby content was created and
    published by individuals, remaining relatively static from that point on.
    Instead, the web became a collaborative platform upon which content could
    be continuously modified by users themselves.

    Though Web 2.0 does not refer to any particular technical ``update'' of the
    web itself, there are certain underlying technologies that are required for
    it to function. Among these are Adobe Flash---allowing videos, audio, and
    other forms of multimedia to be added to web pages---and RSS (Really Simple
    Syndication)---a method for retrieving content that is updated frequently,
    such as blog entries or news headlines.

    User Generated Content refers to, quite simply, content that has been
    generated by the \emph{users} of the web. According to the Organisation for
    Economic Co-operation and Development \cite{vickery2007}, user generated
    content must fulfill three requirements to be considered as such. It must:

    \begin{itemize}
        \item be published on a publicly accessibly website (or social
            networking website accessible to selected users);
        \item show a certain amount of creative effort (i.e. not be a direct
            re-publication of another work);
        \item have been created outside of professional routines and practices.
    \end{itemize}

    Though User Generated Content has been around prior to the rise of Web 2.0,
    the combination of technological, economic and social drivers make it more
    pervasive and different from that which was observed in the first instance.

    Given these definitions, Social Media can be described as a group of
    internet based application that build upon the foundations of Web 2.0,
    allowing the creation and exchange of User Generated Content
    \cite{kaplan2010}.

    New media, meanwhile, has a different, wider meaning. Also, new media is
    a difficult term to define due to the extremely blurred line between
    ``new'' and ``old''. New media captures the development of new technologies
    (most notably the internet), and the remaking of more traditional forms of
    media \cite{flew2008}. The internet is the newest, most significant
    manifestation of new media, allowing information and \emph{knowledge} to be
    published and shared in ways and, indeed, at a speed never before
    possible.

    \section{Mobile Platforms}

    Recent advances in technology have seen content previously only available
    to desktop computers become available to mobile devices also. This owes to
    the increasing power and decreasing size of mobile devices, and the
    widespread availability of mobile internet through 3G and/or WiFi networks.

    Social media on a mobile platform changes the very nature of the media
    itself. Instead of something which is considered a separate task,
    interacting with social media becomes part of one's daily life. The ability
    to access services such as Facebook, Twitter or FourSquare whilst not sat
    at a desk is a nothing short of a revolution in computing history.

    One important aspect of social media, particularly on a mobile platform, is
    that it breaks down boundaries. Whether those boundaries are social,
    economic, geographic or political, social media makes it possible for
    people to interact and debate like never before.

    \section{The Revolution}

    Social media has contributed to change in the world in many ways. Recently,
    however, that has been more true then ever with the protests in North
    Africa and the Middle East.

    The anti-government sentiment which had been simmering in Egypt for some
    time was ignited when protesters used social platforms such as Twitter to
    organise their demonstrations.

    \bibliographystyle{alpha}
    \bibliography{references}

\end{document}
